\chapter{Problesm with Perfect State Information}
In this chapter we consider a number of applications of discrete-tiem stochastic optimal control with perfect state infomation.

\section{Linear Systems and Quadratic Cost}
In this section we consider the special case of a linear Systems
\begin{equation*}
    x_{k + 1} = A_k x_k + B_k u_k + w_k,\quad k = 0,1,\dots,N - 1,
\end{equation*}
and the quadratic cost \begin{equation*}
    \underset{w_k,k = 0,1,\dots,N - 1}{E}\left\{x_N^\prime Q_N x_N +\sum_{k = 0}^{N - 1}(x_k^\prime Q_k x_k + u_k^\prime R_k u_k) \right\}.
\end{equation*}
We assume that $Q_k$ are postive semidefinite symmetric and $R_k$ are positive definite symmetric. $w_k$ has 0 mean and finite second moment.

Applying the DP algorithm, we have 
\begin{align}
    &J_N(x_N) = x_N^\prime Q_N x_N,\nonumber\\
    &J_k(x_k) =\min_{u_k}E\{x_k^\prime Q_k x_k + u_k^\prime R_k u_k + J_{k + 1}(A_k x_k + B_k u_k + w_k)\}.
\end{align}
We have the optimal control law for every $k$:
\begin{equation}\label{eq:3.2}
    \mu_k^*(x_k) = L_k x_k,
\end{equation}
where 
\[L_k = -(B_k^\prime K_{k + 1}B_k + R_k)^{ - 1}B_k^\prime K_{k + 1} A_k,\] 
and where the symmetric positive semidefinite matrices $K_k$ are given by 
\begin{align}
    &K_N = Q_N,\\
    &K_k = A_k^\prime(K_{k + 1} - K_{k + 1}B_k(B_k^\prime K_{k + 1}B_k + R_k)^{ - 1}B_k^\prime K_{k + 1})A_k + Q_k \label{eq:3.4}
\end{align}
The optimal cost is then given by 
\[J_0 (x_0) = x_0^\prime K_0 x_0 + \sum_{k = 0}^{N - 1}E\{w_k^\prime K_{k + 1}w_k\}.\]

\subsection*{The Riccati Equation and Its Asymptotic Behavior}
\cref{eq:3.4} is called the \textit{discrete-time Riccati equation}. If the matrices are constant, then as $k\rightarrow\infty$, $K$ satisfies the \textit{algebraic Riccati equation}
\begin{equation}\label{eq:3.5}
    K = A^\prime (K - KB(B^\prime K B + R)^{ - 1}B^\prime K)A + Q.
\end{equation}
This property indicates that for a large $N$, one can approximate the control law \cref{eq:3.2} by $\{\mu^*,\dots,\mu^*\}$, where
\begin{equation}
    \mu^*(x) = L x,
\end{equation}
\[L = -(B^\prime K B + R)^{ - 1}B^\prime K A,\] and $K$ solves \cref{eq:3.5}. This control law is \textit{stationary}.
\begin{definition}
    A pair $(A, B)$, where $A$ is an $n \times n$ matrix
    and $B$ is an $n \times m$ matrix, is said to be controllable if the $n
    \times n m$ matrix \[ \left[B, A B, A^2 B, \ldots, A^{n-1} B\right] \] has
    full rank (i.e., has linearly independent rows). A pair $(A, C)$, where $A$
    is an $n \times n$ matrix and $C$ an $m \times n$ matrix, is said to be
    observable if the pair $\left(A^{\prime}, C^{\prime}\right)$ is controllable,
    where $A^{\prime}$ and $C^{\prime}$ denote the transposes of $A$ and $C$,
    respectively.
\end{definition}
We have
\begin{align*}
    &x_{k + 1} = Ax_k + Bu_k\\
    \Rightarrow& x_n = A^n x_0 + B u_{n - 1} + AB u_{n - 2} +\cdots + A^{n - 1}Bu_0
\end{align*}
or equivalently
\begin{equation}\label{eq:3.7}
    x_n - A^n x_0 =(B,AB,\dots,A^{n - 1}B)\begin{pmatrix}
        u_{n - 1}\\
        u_{n - 2}\\
        \vdots\\
        u_0
    \end{pmatrix}.
\end{equation}
Since $(A,B)$ is controllable, the right-hand side of \cref{eq:3.7} can be made equal to any vector in $\mathbb{R}^n$. This explains the name of ``controllable pair''. 

\textbf{Observability}: given measurements $z_0,z_1,\dots,z_{n-1}$ of the form $z_k=C x_k$, we can infer the initial state $x_0$ of the system $x_{k+1}=Ax_k$, since 
\[\begin{pmatrix}
    z_{n - 1}\\
    \vdots\\
    z_1\\
    z_0
\end{pmatrix} = \begin{pmatrix}
    CA^{n-1}\\
    \vdots\\
    CA\\
    C
\end{pmatrix}x_0.\]
It's also equivalent to that in the absence of control, if $C x_k\rightarrow 0$ then $x_k\rightarrow 0$.

To simplify notation, we denote $K_{N-k}$ in \cref{eq:3.4} by $P_k$. 
\begin{proposition}
    Let $A$ be an $n\times n$ matrix, $B$ be an $n\times m$ matrix, $Q$ be an $n\times n$ positive semidefinite symmetric matrix, and $R$ be an $m\times m$ positive definite symmetric matrix. Consider 
    \begin{equation}\label{eq:3.8}
        P_{k + 1} = A ^\prime(P_k - P_k B(B^\prime P_k B + R)^{ - 1}B^\prime P_k)A + Q,\quad k = 0,1,\dots,
    \end{equation}
    where $P_0$ is an arbitrary positive semidefinite symmetric matrix. Assume that $(A,B)$ is controllable. Assume also that $Q=C^\prime C$, where $(A,C)$ is observable. Then
    \begin{enumerate}[label=(\alph*)]
        \item $\exists! $ positive definite symmetric matrix $P$ such that for every positive semidefinite symmetric matrix $P_0$ we have \[\lim_{k\rightarrow\infty}P_k = P.\]
        Furthermore, $P$ is the unique solution of \begin{equation}\label{eq:3.9}
            P = A^\prime (P - PB(B^\prime P B + R)^{ - 1}B^\prime P)A + Q
        \end{equation}
        within the class of positive semidefinite symmetric matrices.
        \item The corresponding closed-loop system is stable; i.e., the eigenvalues of the matrix \begin{equation}\label{eq:3.10}
            D = A + B L ,
        \end{equation}
        where \begin{equation}\label{eq:3.11}
            L =- {(B^\prime P B + R)}^{ - 1}B^\prime P A,
        \end{equation}
        are strictly within the unit circle.
    \end{enumerate}
\end{proposition}
\begin{proof}
\textit{Initial Matrix \$P\_0=0\$}. 
    Consider the optimal control problem of finding $u_0,u_1,\dots,u_{k-1}$ that minimize
    \[\sum_{i=0}^{k-1} (x_i^\prime Q x_i + u_i^\prime R u_i)\]
    subject to \[x_{i + 1} = Ax_i + Bu_i,\quad i = 0,1,\dots k - 1,\]
    where $x_0$ is given. The optimal value for this problem, by \cref{eq:3.4}, $x_0^\prime P_k(0) x_0$, is given by \cref{eq:3.8} with $P_0=0 $ (since $K_N=Q_N$). 
    For any control sequence ${(u_0,u_1,\dots,u_{k})} $ we have 
    \[\sum_{i=0}^{k-1} (x_i^\prime Q x_i + u_i^\prime R u_i)\leq \sum_{i=0}^{k} (x_i^\prime Q x_i + u_i^\prime R u_i)\]
    and hence 
    \[\begin{aligned}
        x_0^\prime P_k(0) x_0 &=\min_{u_i,i = 0,\dots,k - 1}\sum_{i=0}^{k-1} (x_i^{\prime} Q x_i + u_i^{\prime} R u_i)\\
            &\leq \min_{u_i,i = 0,\dots,k}\sum_{i=0}^{k}
            (x_i^{\prime} Q x_i + u_i^{\prime} R u_i)\\
            &= x_0^\prime P_{k + 1}(0)x_0,
    \end{aligned}\]
    where both minimizations are subject to the system equation constraint $x_{i+1}=A x_i+B u_i$. Furthermore, for a fixed $x_0$ and for every $k$, $x_0^{\prime} P_k(0) x_0$ is bounded from above by the cost corresponding to a control sequence that forces $x_0$ to the origin in $n$ steps and applies zero control after that. Such a sequence exists by the controllability assumption. Thus the sequence $\{x_0^{\prime} P_k(0) x_0\}$ is nondecreasing with respect to $k$ and bounded from above, and therefore it converges for every $x_0\in\mathbb{R}^n$. Then $P_k(0)$ converges to a $P$ by choosing $x_0=(1,0,\dots,0), (0,1,0,\dots,0),\dots,(1,1,0,\dots,0)$. So we have \[\lim_{k\rightarrow\infty}P_k(0) = P,\]
    where $P$ is generated by \cref{eq:3.8} with $P_0=0$. Then we have \cref{eq:3.9}. 
    By direct calculation we have 
    \begin{equation}\label{eq:3.13}
        P = D^\prime P D + Q + L^\prime R L,
    \end{equation}
    where $D$ and $L$ are given by \cref{eq:3.10} and \cref{eq:3.11}. 

\textit{Stability of the Closed-Loop System.}
    Consider the system \begin{equation}\label{eq:3.15}
        x_{k + 1} =(A + BL)x_k = D x_k
    \end{equation}
    for an arbitrary initial state $x_0$. We will show $x_k\rightarrow 0$. 
    By \cref{eq:3.13}, we have \[x_{k + 1}^\prime P x_{k + 1} - x_k^\prime P x_k = x_k^\prime (D^\prime P D - P)x_k =- x_k^\prime (Q + L^\prime R L)x_k .\]
    hence
    \begin{equation}\label{eq:3.16}
        x_{k + 1}^\prime P x_{k + 1} = x_0^{\prime} P x_0 - \sum_{i=0}^{k }x_i^{\prime} (Q + L^\prime R L)x_i. 
    \end{equation}
    The left-hand side is bounded below by zero, so it follows that 
    \[\lim_{k\rightarrow\infty} x_k^\prime (Q + L^\prime R L)x_k = 0.\]
Since $R$ is positive definite and $Q=C^\prime C$, we have 
\begin{equation}\label{eq:3.17}
    \lim_{k\rightarrow\infty} C x_k = 0,\quad \lim_{k\rightarrow\infty} Lx_k = \lim_{k\rightarrow\infty} \mu^*(x_k) = 0.
\end{equation}

The preceding relations imply that as the control asymptotically becomes negligible, we have $\lim_{k\rightarrow\infty} C x_k=0$, and in view of the observability assumption, this implies that $x_k\rightarrow 0$. To express this argument more precisely, using \cref{eq:3.15}, we have 
\begin{equation}\label{eq:3.18} \left(\begin{array}{c} C\left(x_{k+n-1}-\sum_{i=1}^{n-1}
A^{i-1} B L x_{k+n-i-1}\right) \\ C\left(x_{k+n-2}-\sum_{i=1}^{n-2} A^{i-1} B L
x_{k+n-i-2}\right) \\ \vdots \\ C\left(x_{k+1}-B L x_k\right) \\ C x_k
\end{array}\right)=\left(\begin{array}{c} C A^{n-1} \\ C A^{n-2} \\ \vdots \\ C
A \\ C \end{array}\right) {x_k}.
\end{equation}
Since $Lx_k\rightarrow 0$ by \cref{eq:3.17}, the left-hand side tends to zero and hence the right-hand side tends to zero also. By the observability assumption, the matrix on the right right of \cref{eq:3.18} has full rank to that $x_k\rightarrow 0$.


\textit{Positive Definiteness of $P$.} 
Assume the contrary, i.e., there exists some $x_0\neq 0$ such that $x_0^{\prime} P x_0=0$. Since $P$ is positive semidefinite, from \cref{eq:3.16} we obtain \[x_k^\prime (Q + L^\prime R L)x_k = 0,\quad k = 0,1,\dots\]
Since $x_k\rightarrow 0$, we obtain $x_k^\prime Q x_k=x_k^\prime C^\prime C x_k=0$ and $x_k^\prime L^\prime R L x_k=0$, or
\[Cx_k = 0,\quad L x_k = 0,\quad k = 0,1,\dots\]
Consider \cref{eq:3.18} for $k=0$. By the precreding equalities, we then have 
\[0 = \begin{pmatrix}
    CA^{n - 1}\\ 
    \vdots\\ CA \\ C 
\end{pmatrix}x_0.\]
Then we have $x_0=0$ since the matrix has full rank, which contradicts to the hypothesis $x_0\neq 0$. Thus, $P$ is positive definite.

\textit{Arbitrary Initial Matrix $P_0$.} 


\end{proof}
